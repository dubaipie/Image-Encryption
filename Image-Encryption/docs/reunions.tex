\documentclass{report}

\usepackage[utf8]{inputenc} % un package
\usepackage[T1]{fontenc}      % un second package
\usepackage[francais]{babel}  % un troisième package

\title{Compte-rendus des réunions}
\author{Pierre DUBAILLAY}

\begin{document}
\maketitle

\part*{Semaine du 09/01}
\section*{Introduction}
Cette première réunion a permis de poser les bases du projet, en définissant le cadre, les attentes et les limites attendus.

\section*{Documents demandés}
Les documents suivants sont à rendre à la fin du projet :
\begin{itemize}
	\item[.] une vidéo de démonstration de l'utilisation du programme à des fins publicitaires pour l'université
	\item[.] un rapport de projet, décrivant l'ensemble du projet (des choix d'implentation aux problèmes rencontrés)
	\item[.] le code source du projet
\end{itemize}

\section*{Choix du langage}
Le langage choisit est le python. Il est imposé un code robuste (qui peut faire face à une utilisation non prévue) et maintenable.

\section*{Fonctionnalités imposées}
Le programme doit pouvoir prendre une image dans n'importe quel format et la formatter pour une utilisation. Il doit être possible de fournir une clé pour chiffrer une image.

\section*{Fonctionnalités souhaitées}
Un mode enfant, simple d'accès, et qui permet de dessiner sa propre image.

\section*{DeadLines}
Une première échéance est début février, pour une démonstration aux portes ouvertes de l'université de Rouen.
La seconde échéance est, pour le moment, début avril.

\section*{Structure du programme}
La structure n'est pas encore arrêtée, mais certaines pistes on été levées :
\begin{itemize}
	\item[.] un système de gestion des mises à jour ainsi que des paquets nécessaires
	\item[.] un système de module (plugin), où chaque fonctionnalité (cryptage, génération de clé ...) est dans son propre module. Cela permettrait de mmodulariser le programme, et permettre de nouvelles fonctionnalités non prévues
	\item[.] un représentation des-dits modules sous forme d'onglets
	\item[.] les librairies graphiques et de gestion d'image ne sont pas arrêtées
\end{itemize}

\section*{Réflexions sur les algorithmes}
Il a été soulevé deux interrogations sur les algorithmes :
\begin{itemize}
	\item[.] doit-on diviser chaque pixel en 4 ou bien n'est-ce pas nécessaire
	\item[.] une possibilité d'effectuer l'algorithme XOR avec des couleurs
\end{itemize}

\end{document}
